\documentclass[12pt]{article}

\usepackage{sbc-template}

\usepackage{graphicx,url}

\usepackage[brazil]{babel}   
%\usepackage[latin1]{inputenc}  
\usepackage[utf8]{inputenc}  
% UTF-8 encoding is recommended by ShareLaTex

     
\sloppy

\title{Padrões de Design de Serviços}

\author{João Marcos Ferreira\inst{1}, Kleonte Gomes\inst{2}, Rafael Tavares Rufino\inst{3} }


\address{Instituto Federal de Educação, Ciência e Tecnologia da Paraíba - Campus Cajazeiras (IFPB)\\
  Cajazeiras -- PB -- Brazil
  \email{joaomarccos.ads@gmail.com, kleonte email,
  rafaeltavares.rofino198@gmail.com}
}

\begin{document} 

\maketitle

\begin{abstract}
O objetivo desse artigo é apresentar os principais conceitos de uma Arquitetura orientada a serviços (SOA) e seus seus princípios, bem como padrões de serviços fundamentais apresentados por Thomas Erl em seu livro SOA Design Patterns (2008), que são aplicados na solução de problemas comuns encontrados no desenvolvimento de sistemas baseados em SOA. A união de princípios e padrões de design, formam um conjunto de boas práticas essencial a qualquer projeto.   
\end{abstract}
   
\begin{resumo} 
The objective of this paper is to present the main concepts of a Service-Oriented Architecture (SOA) and its principles, well as Foundational Service Patterns presented by Thomas Erl in his book SOA Design Patterns (2008), which are applied in solving common problems encountered in the development of systems based on SOA. The union of principles and design patterns, form a core set of good practices to any project.
\end{resumo}


\section{O Que é SOA?}

Arquitetura orientada a serviços ou SOA (Service-Oriented Architecture) é descrita de forma diferente por vários autores, alguns o vêem como um estilo técnico de arquitetura que fornece os meios para integrar sistemas distintos e expor as funções de negócios reutilizáveis. Outros autores, no entanto, ter uma visão muito mais ampla:

SOA é uma abordagem arquitetural corporativa que permite a criação de serviços de negócio interoperáveis que podem facilmente ser reutilizados e compartilhados entre aplicações e empresas.

A arquitetura orientada a serviços é um estilo de design que orienta todos os aspectos da criação e utilização de serviços de negócios em todo o seu ciclo de vida (desde a concepção à aposentadoria). %\cite{newcomer}

Arquitetura Orientada a Serviços (SOA) é um paradigma para organização e utilização de capacidades distribuídas que podem estar sob controle de diferentes domínios proprietários. %\cite{oasis}

De maneira simples, SOA é uma abordagem de negócios para criar sistemas de TI (Tecnologia de Informação) que permitem alavancar recursos existentes, criar novos recursos e, principalmente, estar preparado para inevitáveis alterações exigidas pelo mercado, obtendo mais produtividade e lucro para a empresa.


\section{Princípios de Design} 

Os Princípios de design são uma representação do paradigma de design orientado a serviços, eles estipulam boas praticas para construção de um sistema baseado em SOA bem implementado. A seguir estão os princípios de design de serviços listados por %\citeonline{thomas}:

\begin{itemize}
\item Service Abstraction
\item Service Autonomy
\item Service Composability 
\item Service Discoverability
\item Service Loose Coupling
\item Service Reusability
\item Service Statelessness
\item Service-Orientation and Interoperability
\item Standardized Service Contract
\end{itemize}

Especificamente, a potencial relação entre princípios e padrões de projeto orientados a serviço pode ser resumido da seguinte forma:

    Princípios de design são aplicados coletivamente solução lógica, a fim de moldá-la de tal maneira que ela promove as principais características de design que suportam os objetivos estratégicos associados à computação orientada a serviços.
   Os padrões de design fornecem soluções para problemas comuns encontrados ao aplicar os princípios de design e ao estabelecer um ambiente adequado para a execução lógica concebida de acordo com os princípios de orientação a serviços.

\section{Padrões de Design}

No livro SOA Design Patterns de Thomas Erl, são apresentados diversos padrões, contudo serão descritos nesse artigo apenas os chamados padrões de serviços fundamentais, pois esses constituem um conjunto de padrões básicos de design que ajudam a estabelecer as características fundamentais do projeto de serviço através de uma sequência de aplicação sugerida. Coletivamente, esses padrões formam a aplicação mais básica de orientação a serviço.


\subsection{Agnostic Capability}

The subsection titles must be in boldface, 12pt, flush left.

\subsection{Agnostic Context}

The subsection titles must be in boldface, 12pt, flush left.

\subsection{Functional Decomposition}

The subsection titles must be in boldface, 12pt, flush left.

\subsection{Non-Agnostic Context}

The subsection titles must be in boldface, 12pt, flush left.

% if you need to include a figure(eps format), code:
% \begin{figure}[ht]
% \centering
% \includegraphics[width=.5\textwidth]{fig}
% \caption{A typical figure}
% \label{fig:exampleFig1}
% \end{figure}

\section{Considerações Finais}

\bibliographystyle{sbc}
\bibliography{sbc-template}

\end{document}
